\section{Usage}
\subsection{Create configuration file}
As it was mentioned in \textbf{Introduction} in order to configure a digitizer you must
use configuration file for \blt{WD}. The complete instruction how to create it see in
\blt{WD} documentation. This section is how to do this using the GUI presented in this
package. Here there will be no explanation of what each configure option means, for this,
please, check \blt{WD} documentation.

Firstly, open terminal and change to the directory when you want to place config-file.
Then run the GUI there:
\begin{lstlisting}
> caenccf
\end{lstlisting}
You will see the GUI's starting page (see~Fig.\ref{fig:gui_stpg}).
All configure options have the same names as in the official \blt{WD} documentation so
using is straightforward. However, some things require additional explanation. 
\subsubsection*{Pages}
The GUI consists of four pages (including the starting page). The \codet{Common Settings}
page contains configure options applied to each channel. The \codet{Individual Settings}
page is for the channels' configuration. The \codet{Terminal} page contains terminal
frame for running \blt{WD} (see below).
\subsubsection*{Path to file}
It is possible to place config-file in a directory that differs from that in which the
GUI was launched. To save config-file wherever you wish use the \codet{Save to} entry

\begin{figure}[H]
    \centering
    \includegraphics[width=0.4\textwidth]{../pictures/documentation/gui/browser.png}
\end{figure}

By default it is the current directory (\codet{./}) but it can be changed either by
changing the entry directly or by using file browser: \codet{Browse} \inlinegraphics{../pictures/documentation/gui/browser_btn.png} button.

\subsubsection*{Baseline and DC offset}
\begin{figure}[H]
    \centering
    \includegraphics[width=0.4\textwidth]{../pictures/documentation/gui/baseline.png}
\end{figure}
As it is said in \blt{WD} documentation the \codet{BASELINE\tus SHIFT} and
\codet{DC\tus OFFSET} options \emph{are intended to be used one alternatively to the other}.
The GUI reflects this in the following way. If \codet{Use DC offset}
\inlinegraphics{../pictures/documentation/gui/offset.png} check button is OFF then only the
\codet{Baseline shift} scale matters; the \codet{DC offset} scale is disabled and its value
is not shown. And vice-versa --- if that button is ON only
the \codet{DC offset} scale is taken into account, and the \codet{Baseline shift} scale is
disabled.

\subsubsection*{Trigger source option}
There are two choices for trigger: \codet{External} and \codet{Channel}.
\codet{External} trigger can be used either for the acquisition only or for the acquisition-\&-trigger-out signal. \codet{Channel} trigger has one additional option: trigger-out-only.
These options are not independent (for example, it is nonsense to use the \codet{Channel}
trigger for the acquisition and the \codet{External} for the aucquisition-\&-trigger-out
simultaneously). These options are automatically aligned in the interface (the latest changed option has the highest precedence). For example, if you changed the \codet{Channel} trigger option of some channel to the \codet{ACQUISITION\tus ONLY}:
\begin{figure}[H]
    \centering
    \includegraphics[width=0.4\textwidth]{../pictures/documentation/gui/chan.png}
\end{figure}
\noindent then every other channel, if its option is not the \codet{TRGOUT\tus ONLY}, is disabled automatically, so
the \codet{External} trigger:
\begin{figure}[H]
    \centering
    \includegraphics[width=0.4\textwidth]{../pictures/documentation/gui/ext.png}
\end{figure}

\subsubsection*{Saving configuration file} 

Once the configuration is done press \codet{Save} \inlinegraphics{../pictures/documentation/gui/save_btn.png}
button (right lower corner). If succeeded you should see the following popup window
\begin{figure}[H]
    \centering
    \includegraphics[width=0.4\textwidth]{../pictures/documentation/gui/success.png}
\end{figure} 
\noindent and a file called \codet{config.txt} inside the specified directory.

\subsection{Running WaveDump inside GUI}
After successful creation of a config-file it's time to run \blt{WD}.
In the \codet{Terminal} page one will find a frame with running \codet{xterm} inside.
It is intended to eliminate the need for users to use another window to run \blt{WD}.
Go to the \codet{Terminal} page and call \blt{WD} with the path to the config-file you
created before as an argument:
\begin{lstlisting}
> wavedump config.txt
\end{lstlisting}

\Warning{The terminal is running in the directory where the GUI was launched. And if you
chose another directory to save the config-file you must provide the full path to that file.
That is why it is recommended to run the GUI from the directory where you place a config-file.}

In the same page one will find \blt{WD} cheat sheet.  
